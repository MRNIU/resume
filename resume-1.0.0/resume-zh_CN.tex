% !TEX TS-program = xelatex
% !TEX encoding = UTF-8 Unicode
% !Mode:: "TeX:UTF-8"

\documentclass{resume}

\XeTeXlinebreaklocale "zh"
\XeTeXlinebreakskip = 0pt plus 1pt

\begin{document}
\lastupdated
% 不显示页码
\pagenumbering{gobble}

\namesection{牛}{志}{宏}{\urlstyle{same}\href{zhihong@nzhnb.com}{zhihong@nzhnb.com} | 156 6720 0505 | Github/MRNIU | UTC+8}

\section{教育背景}
\datedsubsection{\textbf{西安石油大学} 计算机科学与技术 \textit{工学学士}}{2017.9}

\section{技术能力}
\begin{itemize}[parsep=0.2ex]
  \item \textbf{编程语言}: C/C++, Python, Go, Shell, x86/ARM/RISC-V ASM
  \item \textbf{操作系统与平台}: Linux (内核/嵌入式), QNX, Android, macOS/Windows, AUTOSAR
  \item \textbf{工具链}: Git, Docker, CMake, \LaTeX, CI/CD (Jenkins, GitHub Actions, Travis CI)
  \item \textbf{专业领域}: 嵌入式系统开发、操作系统内核、边缘计算、编译器、计算机图形学
\end{itemize}

\section{项目经历}
\datedsubsection{\textbf{雳泊动力 智能机器人架构} 软件架构工程师}{2025.6 - 至今}
\begin{itemize}
  \item \textbf{负责轮腿/人形机器人电子电气架构及软件系统设计}, 支撑具身智能应用落地。
  \item \textbf{设计并实现基于 Docker 的跨平台(OSX/Win)定制化交叉编译环境}, 显著提升团队开发效率。
  \item \textbf{技术栈}: C/C++, ARM 嵌入式开发, 工业总线 (CAN/EtherCAT), 具身智能框架
\end{itemize}

\datedsubsection{\textbf{ThoughtWorks | 某德系车企 车机中间件开发} 资深开发工程师}{2024.4 - 至今}
\begin{itemize}
  \item \textbf{下一代 QNX+Android 智能座舱中间件开发}, 取代旧有 QNX+RichOS 架构, 实现智能化/娱乐化升级
  \item \textbf{基于高通 VIDC 实现 4K H.264/H.265 超清视频硬解码}, 保障 60fps 高帧率流畅播放
  \item \textbf{设计并实现高性能 SOME/IP 服务发现与调用框架}, 高效打通 QNX-Android 跨系统通信(CAN/ICCSPI)
  \item \textbf{深度优化系统启动流程}: 实现 SA8295 平台 \textbf{2秒冷启动并播放开机动画}(行业领先水平)
  \item \textbf{主导关键问题排查与修复}: 深入分析开机阶段软硬件问题, 推动跨团队协作解决核心缺陷
  \item \textbf{全程把控项目进度与质量}, 确保车机系统按计划高质量量产交付
  \item \textbf{技术栈}: C/C++, ARM 嵌入式开发, SA8295, QNX, AUTOSAR
\end{itemize}

\datedsubsection{\textbf{ThoughtWorks | 某电力行业央企控股集团 边缘计算应用开发} 资深开发工程师}{2023.6 - 2024.4}
\begin{itemize}
  \item \textbf{主导重构高精度低时延电力数据采集系统}, 解决原有系统高成本低效痛点
  \item \textbf{设计并开发支持国标 645-97/645-07/698-45 协议的边缘事件采集框架}
  \item \textbf{技术栈}: C/C++, ARM 嵌入式开发
\end{itemize}

\datedsubsection{\textbf{ThoughtWorks | 某央企控股集团 可观测审计文件系统} 技术负责人 | 负责人}{2021.10 - 2023.6}
\begin{itemize}
  \item \textbf{主导设计并实现信创平台(麒麟/方德/x86/龙芯/ARM64)全链路文件审计系统}, 满足军工级保密要求
  \item \textbf{基于 FUSE 开发内核态文件操作监控模块}, 实时追踪文件读写、复制、加密等敏感行为
  \item \textbf{设计实现 X11 桌面操作审计模块}, 完成用户操作行为全链路追踪
  \item \textbf{实现透明文件加解密引擎}, 保障涉密文件零感知安全防护
  \item \textbf{技术栈}: Golang, C/C++, FUSE, X11, Linux Kernel, GTK
\end{itemize}

\datedsubsection{\textbf{ThoughtWorks | 国内某智能穿戴设备研发} 资深开发工程师}{2021.9 - 2021.10}
\begin{itemize}
  \item \textbf{智能穿戴设备应用架构重构}: 针对产品线扩展后 C++ MVP 架构的维护困难、冲突多、开发慢问题, 设计并实现基于 MVVM 架构的应用开发框架
  \item \textbf{核心工作}: 应用重构、前端(HTML/CSS)开发、后端数据联调、框架组件化与功能迭代、缺陷修复
  \item 与框架团队交流, 催熟组件, bug修复
  \item \textbf{技术栈}: C++, Open HarmonyOS, ARM 嵌入式开发
\end{itemize}

\datedsubsection{\textbf{Google Summer of Code | 为 Arduino Portenta H7 移植 FreeRTOS} 负责人}{2020.6 - 2020.9}
\begin{itemize}
  \item \textbf{主导将 FreeRTOS 移植至 Arduino Portenta H7 (非对称多核架构)}, 充分挖掘硬件潜力
  \item \textbf{与硬件、软件、市场团队协作}, 确保 FreeRTOS 与 Arduino Framework 的兼容性支持
  \item \textbf{技术栈}: C/C++, FreeRTOS, Arduino Framework
\end{itemize}

\datedsubsection{\textbf{开源软件供应链点亮计划 | Simple-XX} 社区负责人 | 学生导师}{2020.6 - 至今}
\begin{itemize}
  \item \textbf{负责社区管理与对外联络}(中科院软件所/openEuler社区)
  \item \textbf{指导学生完成社区项目开发工作}
\end{itemize}

% 续页提示
\continue
% 换页
\clearpage

\datedsubsection{\textbf{Alibaba Summer of Code | 为 AliOS Things 实现对 Arduino 框架的支持} 负责人}{2019.6 - 2019.9}
\begin{itemize}
  \item \textbf{从 400+ 申请者中脱颖而出}, 成为 21 位入选者之一
  \item \textbf{主导为 AliOS-Things 物联网操作系统适配 Arduino 框架及标准库}, 解决其原生接口生态受限问题
  \item \textbf{核心工作}: 完成 HAL 层移植、Arduino 框架核心功能支持、标准库集成
  \item \textbf{技术栈}: C/C++, Embedded Kernel (AliOS Things)
\end{itemize}

\datedsubsection{\textbf{基于机器学习技术的数学公式识别研究} 负责人}{2019.12 - 2020.6}
\begin{itemize}
  \item \textbf{使用卷积神经网络(CNN)实现数学公式识别}
  \item \textbf{独立完成应用原型(APP)设计、开发与实现}, 集成公式识别、跨平台数据同步功能
  \item \textbf{负责核心接口封装与 RESTful API 设计}
\end{itemize}

\datedsubsection{\textbf{Simple-XX 社区} 发起人}{2019.12 - 2020.6}
\begin{itemize}
  \item \textbf{创立并运营专注于底层基础软件(处理器/编译器/操作系统/图形学/物理仿真)的开发者社区}
  \item \textbf{负责社区战略规划、对外事务处理及日常运营维护}, 营造积极技术氛围
\end{itemize}

\section{开源贡献}
\begin{tabular}{ll}
  \textbf{U-Boot} & 修复多起 RISCV64/ARM 架构下内核启动问题(ELF 加载、参数传递、FIT 启动)\\
  \textbf{Homebrew/homebrew-core} & 内核开发相关工具链 \\
  \textbf{Simple-XX/SimpleKernel} & riscv/arm/x64 操作系统内核 \\
  \textbf{Simple-XX/SimpleCompiler} & C-sub 编译器 \\
  \textbf{Simple-XX/SimpleRenderer} & 软件光栅渲染器 \\
  \textbf{Simple-XX/SimplePhysicsEngine} & 流体模拟物理引擎 \\
  \textbf{EugeneLiu/translationCSAPP} & CMU 15-213 课程视频翻译计划, 参与翻译校对 \\
  \textbf{xitu/gold-miner} & 参与英文技术文章、tensorflow 官方文档的翻译校对 \\
  More... & 见 Github/MRNIU
\end{tabular}
\sectionsep

\section{所获奖项/其它} 
\begin{tabular}{rcl}
    2024         & 指导学生完成 SimpleRenderer 渲染功能开发 & 开源软件供应链点亮计划 \\
    2023         & 指导学生完成 SimpleKernel UEFI 启动支持 & 开源软件供应链点亮计划 \\
    2021         & 优胜奖 & 全国大学生系统能力大赛 \\ 
    2021         & 拍易得软件 & 软件著作权 \\
    2021         & 指导学生完成 SimpleCompiler 编译器架构 & 开源软件供应链点亮计划 \\
    2021         & 指导学生完成 SimpleKernel RISCV 架构多核支持 & 开源软件供应链点亮计划 \\
    2021         & 指导学生完成 Simple 项目文档编写 & 开源软件供应链点亮计划 \\
    2020         & 指导学生完成 SimpleKernel 开发文档编写 & 开源软件供应链点亮计划 \\
    2020         & 指导学生完成 SimpleKernel 多任务调度系统 & 开源软件供应链点亮计划 \\
    2020         & 指导学生完成 SimpleKernel 内存管理优化 & 开源软件供应链点亮计划 \\
    2020         & 优秀开发者 & openEuler 高校开发者大赛 \\ 
    2020         & 讲者 & 中文学生开源年会 \\
    2019         & LaTeXocr 软件 & 软件著作权 \\
    2019         & 基于机器学习技术的数学公式识别研究 & 国家级创新创业训练计划项目 \\
    2019         & 银奖    & 第五届中国“互联网+”大学生创新创业大赛陕西赛区复赛 \\
    2018         & 二等奖  & 第十二届 iCAN 国际创新创业大赛中国总决赛 \\
\end{tabular}
\sectionsep

\end{document}
