% !TEX TS-program = xelatex
% !TEX encoding = UTF-8 Unicode
% !Mode:: "TeX:UTF-8"

\documentclass{resume}

\XeTeXlinebreaklocale "zh"
\XeTeXlinebreakskip = 0pt plus 1pt

\begin{document}
\lastupdated
% 不显示页码
\pagenumbering{gobble}

\namesection{牛}{志}{宏}{\urlstyle{same}\href{zone.niuzh@hotmail.com}{zone.niuzh@hotmail.com} | 156 6720 0505 | Github/MRNIU | UTC+8}

\section{教育背景}
\datedsubsection{\textbf{西安石油大学} 计算机科学与技术 \textit{工学学士}}{2017.9}

\section{技术能力}
\begin{itemize}[parsep=0.2ex]
  \item \textbf{编程语言}: C/C++,Python,Go,Shell,x86 ASM,ARM ASM,RISCV ASM
  \item \textbf{操作系统与工具}: Linux/macOS/Windows/QNX,Git,Docker,CMake,\LaTeX
  \item \textbf{DevOps}: Travis CI,GitHub Action,Jenkins
  \item \textbf{领域}: Embedded System,Kernel,Edge Computing,Compiler,Computer Graphics
\end{itemize}

\section{项目经历}
\datedsubsection{\textbf{ThoughtWorks | 某德系车企 车机中间件开发} 资深开发工程师}{2024.4 - 至今}
\begin{itemize}
  \item 该企业旧有车载系统为 QNX+RichOS 架构,不能适应以智能化、娱乐化为主的发展方向。本项目使用 QNX+Android 的技术路线进行开发。
  \item 负责开机阶段的启动流程,确保 vCPU+BSP+QNX+Android 之间的 CAN、ICCSPI、SOMEIP 等链路正常工作。
  \item 负责开机阶段的性能优化,保证在 SA8295 上电两秒内可以播放开机动画。
  \item 技术栈 C,C++,ARM,SA8295,QNX,AUTOSAR
\end{itemize}

\datedsubsection{\textbf{ThoughtWorks | 某电力行业央企控股集团 边缘计算应用开发} 资深开发工程师}{2023.6 - 2024.4}
\begin{itemize}
  \item 该企业已有的开发流程成本高、效率低,严重影响企业的生产流程,本项目的目标是结合敏捷实践降本增效,带领企业团队熟悉现代开发方式。
  \item 负责实现电力行业标准 645-97/645-07/698-45 协议的事件采集相关功能。
  \item 负责大规模高准度低时延采集系统架构设计。
  \item 技术栈 C,C++,ARM
\end{itemize}

\datedsubsection{\textbf{ThoughtWorks | 某央企控股集团 可观测审计文件系统} 资深开发工程师 | 负责人}{2021.10 - 2023.6}
\begin{itemize}
  \item 该项目主要是针对信创平台(中科方德、银河麒麟、x86、龙芯 MIPS、ARM64)实现基于 fuse 的可观测审计系统。
  \item 该集团业务有严格的保密需求,要求监控文件的打开、关闭、读写、复制粘贴等操作,对加密文件需要支持用户无感知的透明加解密方案。
  \item 参与项目的全流程,包括需求评审、设计评审,制定测试计划,设计和执行测试用例,及测试回归,进行缺陷跟踪和软件质量分析等。
  \item 负责实现基于 fuse 的用户态文件系统监控审计全功能开发。
  \item 负责实现基于 X11 的用户操作监控功能。
  \item 负责实现用户操作审计。
  \item 负责实现文件系统透明加解密功能。
  \item 技术栈 Golang,C,C++,fuse,X11,linux,systemd,gtk,wps-plugins
\end{itemize}

\datedsubsection{\textbf{ThoughtWorks | 国内某智能穿戴设备研发} 资深开发工程师}{2021.9 - 2021.10}
\begin{itemize}
  \item 该项目主要是对某国内厂商已有的穿戴设备上的应用进行重构。
  \item 在以往的开发中,此厂商主要采用 c++ 和 mvp 架构来开发自己的手表应用,但是随着产品的增多,
    这种架构面临着维护困难,冲突多,开发缓慢等问题。为此开发了一套 MVVM 架构的手表应用开发框架,
    页面基于 html/css 开发,后端封装可复用的组件,整个项目需要不断的催熟框架以及重构应用。
  \item 负责应用的重构,前端编写,以及后端数据联调。
  \item 与框架团队交流,催熟组件,bug修复。
  \item 技术栈 C++,Open HarmonyOS
\end{itemize}

\datedsubsection{\textbf{Google Summer of Code | 为 Arduino Portenta H7 移植 FreeRTOS} 负责人}{2020.6 - 2020.9}
\begin{itemize}
  \item Portenta H7 是 Arduino 推出的一款非对称架构高性能处理器,为了充分发挥硬件性能,社区尝试在 H7 上运行 FreeRTOS。
  \item 与硬件、软件、市场等多个部门合作,负责实现对 FreeRTOS 的移植与对 Arduino Framework 框架的支持。
  \item 技术栈 C,C++,FreeRTOS,Arduino Framework
\end{itemize}

\datedsubsection{\textbf{开源软件供应链点亮计划 | Simple-XX} 社区负责人 | 学生导师}{2020.6 - 至今}
\begin{itemize}
  \item 由中科院软件所与 openEuler 社区主办,负责与社区与主办方的交流,指导学生完成开发工作。
\end{itemize}

% 续页提示
\continue
% 换页
\clearpage

\datedsubsection{\textbf{Alibaba Summer of Code | 为 AliOS Things 实现对 Arduino 框架的支持} 负责人}{2019.6 - 2019.9}
\begin{itemize}
  \item 首届 ASoC,共有 400+ 位申请学生,其中 21 位被阿里巴巴接收。
  \item 与社区紧密合作,为 AliOS-Things 实现对 Arduino 框架及其标准库的支持。
  \item AliOS Things 是 AliOS 家族旗下、面向 IoT 领域的、高可伸缩的物联网操作系统。
  \item 在本项目前,Things 维护了一套自己的接口,不利于大范围推广。
  \item 针对该痛点调研市场状况,为与 Things 配套的 Developer Kit 开发板量身定制了 Things-Arduino 方案。
  \item 负责实现对 Things-Arduino HAL 的移植。
  \item 负责实现对 Arduino 框架的支持。
  \item 负责实现对 Arduino 标准库的支持。
  \item 技术栈 C,C++,Embedded Kernel
\end{itemize}

\datedsubsection{\textbf{基于机器学习技术的数学公式识别研究} 负责人}{2019.12 - 2020.6}
\begin{itemize}
  \item 使用卷积神经网络实现数学公式识别。
  \item 负责从无到有的设计,研发,流程图及开发文档,用两个月的时间独立开发了初版的 APP,集成了公式识别、数据跨平台同步等功能。
  \item 作为软件开发团队的核心成员,为 APP 开发提供高质量的接口封装。对 RESTful 有实践经验。
\end{itemize}

\datedsubsection{\textbf{Simple-XX 社区} 发起人}{2019.12 - 2020.6}
\begin{itemize}
  \item 专注于底层基础软件,包括处理器、编译器、操作系统、计算机图形学、物理仿真等。
  \item 规划社区发展方向,处理社区对外事务。
  \item 维持社区运行,保证社区良好氛围。
\end{itemize}

\section{开源贡献}
\begin{tabular}{ll}
  Homebrew/homebrew-core & 内核开发相关工具链 \\
  Simple-XX/SimpleKernel & riscv/arm/x64 操作系统内核 \\
  Simple-XX/SimpleCompiler & C-sub 编译器 \\
  Simple-XX/SimpleRenderer & 软件光栅渲染器 \\
  Simple-XX/SimplePhysicsEngine & 流体模拟物理引擎 \\
  EugeneLiu/translationCSAPP & CMU 15-213 课程视频翻译计划,参与翻译校对 \\
  xitu/gold-miner & 参与英文技术文章、tensorflow 官方文档的翻译校对 \\
  More... & 见 Github/MRNIU
\end{tabular}
\sectionsep

\section{所获奖项/其它} 
\begin{tabular}{rcl}
    2024         & 指导学生完成 SimpleRenderer 渲染功能开发 & 开源软件供应链点亮计划 \\
    2023         & 指导学生完成 SimpleKernel UEFI 启动支持 & 开源软件供应链点亮计划 \\
    2021         & 优胜奖 & 全国大学生系统能力大赛 \\ 
    2021         & & 软件著作权 \\
    2021         & 指导学生完成 SimpleCompiler 编译器架构 & 开源软件供应链点亮计划 \\
    2021         & 指导学生完成 SimpleKernel RISCV 架构多核支持 & 开源软件供应链点亮计划 \\
    2021         & 指导学生完成 Simple 项目文档编写 & 开源软件供应链点亮计划 \\
    2020         & 指导学生完成 SimpleKernel 开发文档编写 & 开源软件供应链点亮计划 \\
    2020         & 指导学生完成 SimpleKernel 多任务调度系统 & 开源软件供应链点亮计划 \\
    2020         & 指导学生完成 SimpleKernel 内存管理优化 & 开源软件供应链点亮计划 \\
    2020         & 优秀开发者 & openEuler 高校开发者大赛 \\ 
    2020         & & 开源软件供应链 2020 峰会 \\
    2020         & & 中文学生开源年会 \\
    2019         & & 软件著作权 \\
    2019         & & 国家级创新创业训练计划项目 \\
    2019         & 银奖    & 第五届中国“互联网+”大学生创新创业大赛陕西赛区复赛 \\
    2018	       & 二等奖  & 第十二届 iCAN 国际创新创业大赛中国总决赛 \\
\end{tabular}
\sectionsep

\end{document}
