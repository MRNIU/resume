%%%%%%%%%%%%%%%%%%%%%%%%%%%%%%%%%%%%%%%
% Deedy - One Page Two Column Resume
% LaTeX Template
% Version 1.2 (16/9/2014)
%
% Original author:
% Debarghya Das (http://debarghyadas.com)
%
% Original repository:
% https://github.com/deedydas/Deedy-Resume
%
% IMPORTANT: THIS TEMPLATE NEEDS TO BE COMPILED WITH XeLaTeX
%
% This template uses several fonts not included with Windows/Linux by
% default. If you get compilation errors saying a font is missing, find the line
% on which the font is used and either change it to a font included with your
% operating system or comment the line out to use the default font.
% 
%%%%%%%%%%%%%%%%%%%%%%%%%%%%%%%%%%%%%%
% 
% TODO:
% 1. Integrate biber/bibtex for article citation under publications.
% 2. Figure out a smoother way for the document to flow onto the next page.
% 3. Add styling information for a "Projects/Hacks" section.
% 4. Add location/address information
% 5. Merge OpenFont and MacFonts as a single sty with options.
% 
%%%%%%%%%%%%%%%%%%%%%%%%%%%%%%%%%%%%%%
%
% CHANGELOG:
% v1.1:
% 1. Fixed several compilation bugs with \renewcommand
% 2. Got Open-source fonts (Windows/Linux support)
% 3. Added Last Updated
% 4. Move Title styling into .sty
% 5. Commented .sty file.
%
%%%%%%%%%%%%%%%%%%%%%%%%%%%%%%%%%%%%%%%
%
% Known Issues:
% 1. Overflows onto second page if any column's contents are more than the
% vertical limit
% 2. Hacky space on the first bullet point on the second column.
%
%%%%%%%%%%%%%%%%%%%%%%%%%%%%%%%%%%%%%%


\documentclass[]{deedy-resume-openfont}
\usepackage{fancyhdr}
    
\pagestyle{fancy}
\fancyhf{}
    
\begin{document}

%%%%%%%%%%%%%%%%%%%%%%%%%%%%%%%%%%%%%%
%
%     LAST UPDATED DATE
%
%%%%%%%%%%%%%%%%%%%%%%%%%%%%%%%%%%%%%%
\lastupdated

%%%%%%%%%%%%%%%%%%%%%%%%%%%%%%%%%%%%%%
%
%     TITLE NAME
%
%%%%%%%%%%%%%%%%%%%%%%%%%%%%%%%%%%%%%%
\namesection{牛}{志}{宏}{ \urlstyle{same}\href{zone.niuzh@hotmail.com}{@zone.niuzh} | 156 6720 0505 | UTC+8
}

%%%%%%%%%%%%%%%%%%%%%%%%%%%%%%%%%%%%%%
%
%     COLUMN ONE
%
%%%%%%%%%%%%%%%%%%%%%%%%%%%%%%%%%%%%%%

\begin{minipage}[t]{0.25\textwidth}

%%%%%%%%%%%%%%%%%%%%%%%%%%%%%%%%%%%%%%
%     EDUCATION
%%%%%%%%%%%%%%%%%%%%%%%%%%%%%%%%%%%%%%

\section{教育经历}
\subsection{西安石油大学}
\descript{学士学位,主修计算机科学}
\location{2017.09-2022.06}
\sectionsep

%%%%%%%%%%%%%%%%%%%%%%%%%%%%%%%%%%%%%%
%     LINKS
%%%%%%%%%%%%%%%%%%%%%%%%%%%%%%%%%%%%%%

\section{链接}
Github:// \href{https://github.com/MRNIU}{\bf MRNIU} \\
(120+ 关注者) \\
LinkedIn://  \href{https://www.linkedin.com/in/zone-niuzh}{\bf zone-niuzh} \\
\sectionsep

%%%%%%%%%%%%%%%%%%%%%%%%%%%%%%%%%%%%%%
%     COURSEWORK
%%%%%%%%%%%%%%%%%%%%%%%%%%%%%%%%%%%%%%

% \section{修读课程}
% \subsection{Graduate}
% Advanced Machine Learning \\
% Open Source Software Engineering \\
% Advanced Interactive Graphics \\
% Compilers + Practicum \\
% Cloud Computing \\
% Evolutionary Computation \\
% Defending Computer Networks \\
% Machine Learning \\
% \sectionsep

%%%%%%%%%%%%%%%%%%%%%%%%%%%%%%%%%%%%%%
%     SKILLS
%%%%%%%%%%%%%%%%%%%%%%%%%%%%%%%%%%%%%%

\section{技能}
\subsection{编程}
\location{超过 5000 行}
C \textbullet{} C++ \textbullet{} Python \textbullet{} Go \\
\location{1000 - 5000 行}
Scheme \textbullet{} Java \\
\location{低于 1000 行}
Shell \textbullet{} \LaTeX \textbullet{} x86 ASM \\
ARM ASM \textbullet{} RISCV ASM \\
\sectionsep

\subsection{平台}
\location{一般}
x86/x86\_64 \\
ARM/AARCH64 \\
\location{了解}
RISCV/RISCV64 \\
\sectionsep

\subsection{DevOps}
\location{一般}
Git \textbullet{} GitHub Action \\
CMake \textbullet{} Travis CI

熟悉c/c++/python/shell编程以及相关工具 (cmake, makefile, gdb, git 等)
了解内核实现原理,发起开源项目 SimpleKernel 并持续维护
了解渲染器实现原理,发起开源项目 SimpleRenderer 并持续维护
了解 Arduino 及其 Framework
了解 FreeRTOS 实现原理
熟悉 linux 知识,能使用 linux 作为日常开发环境
领域:嵌入式、内核、社区运营

%%%%%%%%%%%%%%%%%%%%%%%%%%%%%%%%%%%%%%
%
%     COLUMN TWO
%
%%%%%%%%%%%%%%%%%%%%%%%%%%%%%%%%%%%%%%

\end{minipage} 
\hfill
\begin{minipage}[t]{0.73\textwidth} 

\section{项目经历}
\runsubsection{阿里巴巴编程之夏}
\descript{学生参与者}
\location{2019.06 - 2019.09 | 远程}
\vspace{\topsep}
\begin{tightemize}
    \item 首届 ASoC,共有 400+ 位申请学生,其中 21 位被阿里巴巴接收
    \item 与社区紧密合作,为 AliOS-Things 实现对 Arduino 框架及其标准库的支持
\end{tightemize}
\sectionsep

\runsubsection{谷歌编程之夏}
\descript{学生参与者}
\location{2020.06 - 2020.08 | 远程}
\begin{tightemize}
\item 为非对称架构的 Portenta H7 Board 移植了 FreeRTOS,以及基于 FreeRTOS 的 Arduino Framework
\item \href{https://github.com/MRNIU/FreeRTOS-PortentaH7}{FreeRTOS-PortentaH7} 为 Portenta H7 移植了 FreeRTOS
\item \href{https://github.com/MRNIU/ArduinoCore-freertos}{ArduinoCore-freertos} 在 FreeRTOS 的基础上编写了 Arduino 编程框架
\end{tightemize}
\sectionsep

\runsubsection{开源软件供应链点亮计划}
\descript{导师}
\location{2020.05 - 2020.09 | 远程}
\begin{tightemize}
\item 由中科院软件所与 openEuler 社区主办,指导学生完成开发工作
\end{tightemize}
\sectionsep

\runsubsection{谷歌编程之夏}
\descript{学生参与者}
\location{2022.06 - 至今 | 远程}
\begin{tightemize}
\item 为 \href{https://www.haiku-os.org}{Haiku} 操作系统进行 ARM 平台的移植,以及基于 Device Tree 的设备驱动
\item \href{https://summerofcode.withgoogle.com/programs/2022/projects/y2L1QWf1}{GSoC2022 中选信息}
\end{tightemize}
\sectionsep

\runsubsection{ThoughtWorks}
\descript{开发实习生}
\location{2020.09 - 至今 | 西安}
\begin{tightemize}
\item 安全与系统研发部门
\item 某品牌智能手表 MVVM 改造
\item 基于 FUSE 的文件监控系统
\end{tightemize}
\sectionsep

\runsubsection{\href{https://github.com/Simple-XX}{\bf Simple-XX 社区}}
\descript{创始人}
\location{2017.09}
\begin{tightemize}
    \item 面向初学者提供结构化的代码,旨在降低学习门槛,助力学生发展
    \item 开源软件供应链点亮计划合作社区
    \item 拥有 CPU、内核、编译器、渲染器等多个的项目
    \item 获得 2k+ star
    \end{tightemize}
\sectionsep

\runsubsection{国家级创新创业训练计划项目}
\descript{负责人}
\location{2019.12}
\begin{tightemize}
    \item 基于机器学习技术的数学公式识别研究
    \end{tightemize}
\sectionsep

\runsubsection{为 AliOS Things 实现对 Arduino 框架的支持}
\descript{负责人}
\location{2019.6}
\begin{tightemize}
    \item 基于机器学习技术的数学公式识别研究
    项目概述
    项目概述
    AliOS Things 是 AliOS 家族旗下、面向 IoT 领域的、高可伸缩的物联网操作系统。在本项目前,Things 维护了一套自己的接口,不利于大范围推广。本项目针对该痛点调研市场状况,为与 Things 配套的 Developer Kit 开发板量身定制了 Things-Arduino 方案。
    项目职责
    负责实现对 Things-Arduino HAL 的移植
    负责实现对 Arduino 框架的支持
    负责实现对 Arduino 标准库的支持
    技术栈
    C
    C++
    嵌入式操作系统
\end{tightemize}
\sectionsep

\runsubsection{为 Arduino Portenta H7 移植 FreeRTOS}
\descript{负责人}
\location{2020.6}
\begin{tightemize}
    \item 基于机器学习技术的数学公式识别研究
    项目概述
    Portenta H7 是 Arduino 推出的一款非对称架构高性能处理器。为了充分发挥硬件性能,社区尝试在 H7 上运行 FreeRTOS。
    项目职责
    负责实现对 FreeRTOS 的移植
    负责实现对 Arduino Framework 框架的支持
    技术栈
    C
    C++
    FreeRTOS
    Arduino Framework
\end{tightemize}
\sectionsep

\runsubsection{国内某智能穿戴设备研发}
\descript{负责人}
\location{2021.9}
\begin{tightemize}
    \item 基于机器学习技术的数学公式识别研究
    项目概述
    该项目主要是对某国内厂商已有的穿戴设备上的应用进行重构。在以往的开发中,客户主要采用c++和mvp架构来开发自己的手表应用,但是随着产品的增多,这种架构面临着维护困难,冲突多,开发缓慢等特点。为此,开发了一套mvvm架构的手表应用开发框架,页面基于html/css开发,后端封装可复用的组件,整个项目需要不断的催熟框架以及重构应用。
    项目职责
    应用的重构,前端编写,以及后端数据联调。
    组件催熟,bug修复
    技术栈
    C++
    LiteOS
\end{tightemize}
\sectionsep

\runsubsection{某央企控股集团 可观测审计文件系统}
\descript{负责人}
\location{2021.10}
\begin{tightemize}
    项目概述
    该项目主要是针对信创平台(中科方德、银河麒麟、x86、龙芯 MIPS、ARM64)实现基于 fuse 的可观测审计系统。该集团业务有严格的保密需求,要求监控文件的打开、关闭、读写、复制粘贴等操作,对加密文件需要支持用户无感知的透明加解密方案。
    项目职责
    负责实现基于 fuse 的用户态文件系统监控审计全功能开发
    负责实现基于 X11 的用户操作监控功能
    负责实现用户操作审计
    负责实现文件系统透明加解密功能
    技术栈
    Golang
    C
    fuse
    X11
\end{tightemize}
\sectionsep


\runsubsection{某电力行业央企控股集团 边缘计算应用开发}
\descript{负责人}
\location{2023.6}
\begin{tightemize}
    项目概述
    该企业已有的开发流程成本高、效率低,严重影响企业的生产流程,本项目的目标是结合敏捷实践降本增效,带领企业团队熟悉现代开发方式。
    项目职责
    负责实现电力行业标准 698 协议的事件采集相关功能
    bug 修复
    大规模采集的流程情况的性能优化
    技术栈
    C
    ARM
\end{tightemize}
\sectionsep

%%%%%%%%%%%%%%%%%%%%%%%%%%%%%%%%%%%%%%
%     OPEN SOURCE
%%%%%%%%%%%%%%%%%%%%%%%%%%%%%%%%%%%%%%

\section{开源贡献}
\begin{tabular}{ll}
    \href{https://github.com/EugeneLiu/translationCSAPP}{EugeneLiu/translationCSAPP} & CMU 15-213 课程视频翻译计划,参与翻译校对 \\
    \href{https://github.com/xitu/gold-miner}{xitu/gold-miner} & 参与英文技术文章、tensorflow 官方文档的翻译校对 \\
    \href{https://github.com/Homebrew/homebrew-core}{Homebrew/homebrew-core} & 内核开发相关工具链 \\
    More... & 见 \href{https://github.com/MRNIU}{Github/MRNIU}
\end{tabular}
\sectionsep

%%%%%%%%%%%%%%%%%%%%%%%%%%%%%%%%%%%%%%
%     AWARDS
%%%%%%%%%%%%%%%%%%%%%%%%%%%%%%%%%%%%%%

\section{所获奖项} 
\begin{tabular}{rll}
    2019         & 银奖    & 第五届中国“互联网+”大学生创新创业大赛陕西赛区复赛 \\
    2018	     & 二等奖  & 第十二届 iCAN 国际创新创业大赛中国总决赛 \\
\end{tabular}
\sectionsep

%%%%%%%%%%%%%%%%%%%%%%%%%%%%%%%%%%%%%%
%     PUBLICATIONS
%%%%%%%%%%%%%%%%%%%%%%%%%%%%%%%%%%%%%%

% \section{Publications} 
% \renewcommand\refname{\vskip -1.5cm} % Couldn't get this working from the .cls file
% \bibliographystyle{abbrv}
% \bibliography{publications}
% \nocite{*}

\end{minipage} 
\end{document}  \documentclass[]{article}
