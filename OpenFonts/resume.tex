%%%%%%%%%%%%%%%%%%%%%%%%%%%%%%%%%%%%%%%
% Deedy - One Page Two Column Resume
% LaTeX Template
% Version 1.2 (16/9/2014)
%
% Original author:
% Debarghya Das (http://debarghyadas.com)
%
% Original repository:
% https://github.com/deedydas/Deedy-Resume
%
% IMPORTANT: THIS TEMPLATE NEEDS TO BE COMPILED WITH XeLaTeX
%
% This template uses several fonts not included with Windows/Linux by
% default. If you get compilation errors saying a font is missing, find the line
% on which the font is used and either change it to a font included with your
% operating system or comment the line out to use the default font.
% 
%%%%%%%%%%%%%%%%%%%%%%%%%%%%%%%%%%%%%%
% 
% TODO:
% 1. Integrate biber/bibtex for article citation under publications.
% 2. Figure out a smoother way for the document to flow onto the next page.
% 3. Add styling information for a "Projects/Hacks" section.
% 4. Add location/address information
% 5. Merge OpenFont and MacFonts as a single sty with options.
% 
%%%%%%%%%%%%%%%%%%%%%%%%%%%%%%%%%%%%%%
%
% CHANGELOG:
% v1.1:
% 1. Fixed several compilation bugs with \renewcommand
% 2. Got Open-source fonts (Windows/Linux support)
% 3. Added Last Updated
% 4. Move Title styling into .sty
% 5. Commented .sty file.
%
%%%%%%%%%%%%%%%%%%%%%%%%%%%%%%%%%%%%%%%
%
% Known Issues:
% 1. Overflows onto second page if any column's contents are more than the
% vertical limit
% 2. Hacky space on the first bullet point on the second column.
%
%%%%%%%%%%%%%%%%%%%%%%%%%%%%%%%%%%%%%%


\documentclass[]{deedy-resume-openfont}
\usepackage{fancyhdr}
    
\pagestyle{fancy}
\fancyhf{}
    
\begin{document}

%%%%%%%%%%%%%%%%%%%%%%%%%%%%%%%%%%%%%%
%
%     LAST UPDATED DATE
%
%%%%%%%%%%%%%%%%%%%%%%%%%%%%%%%%%%%%%%
\lastupdated

%%%%%%%%%%%%%%%%%%%%%%%%%%%%%%%%%%%%%%
%
%     TITLE NAME
%
%%%%%%%%%%%%%%%%%%%%%%%%%%%%%%%%%%%%%%
\namesection{Zhihong}{Niu}{ \urlstyle{same}\href{zone.niuzh@hotmail.com}{@zone.niuzh} | 156 6720 0505 | UTC+8
}

%%%%%%%%%%%%%%%%%%%%%%%%%%%%%%%%%%%%%%
%
%     COLUMN ONE
%
%%%%%%%%%%%%%%%%%%%%%%%%%%%%%%%%%%%%%%

\begin{minipage}[t]{0.25\textwidth} 

%%%%%%%%%%%%%%%%%%%%%%%%%%%%%%%%%%%%%%
%     EDUCATION
%%%%%%%%%%%%%%%%%%%%%%%%%%%%%%%%%%%%%%

\section{Education} 

\subsection{Xi'an Shiyou University}
\descript{BE in Computer Science}
\location{2017.09-2022.09}
\sectionsep

%%%%%%%%%%%%%%%%%%%%%%%%%%%%%%%%%%%%%%
%     LINKS
%%%%%%%%%%%%%%%%%%%%%%%%%%%%%%%%%%%%%%

\section{Links}
Github:// \href{https://github.com/MRNIU}{\bf MRNIU} \\
(120+ followers) \\
LinkedIn://  \href{https://www.linkedin.com/in/zone-niuzh}{\bf zone-niuzh} \\
\sectionsep

%%%%%%%%%%%%%%%%%%%%%%%%%%%%%%%%%%%%%%
%     COURSEWORK
%%%%%%%%%%%%%%%%%%%%%%%%%%%%%%%%%%%%%%

%%%%%%%%%%%%%%%%%%%%%%%%%%%%%%%%%%%%%%
%     SKILLS
%%%%%%%%%%%%%%%%%%%%%%%%%%%%%%%%%%%%%%

\section{Skills}
\subsection{Programming}
\location{Over 5000 lines}
C \textbullet{} C++ \textbullet{} Python \textbullet{} Go \\
\location{1000 - 5000 lines}
Scheme \textbullet{} Java \\
\location{Less than 1000 lines}
Shell \textbullet{} \LaTeX \textbullet{} x86 ASM \\
ARM ASM \textbullet{} RISCV ASM \\
\sectionsep

\subsection{Platform}
\location{Familiar}
x86/x86\_64 \\
ARM/AARCH64 \\
\location{Knowledge}
RISCV/RISCV64 \\
\sectionsep

\subsection{DevOps}
\location{Familiar}
Git \textbullet{} GitHub Action \\
CMake \textbullet{} Travis CI

%%%%%%%%%%%%%%%%%%%%%%%%%%%%%%%%%%%%%%
%
%     COLUMN TWO
%
%%%%%%%%%%%%%%%%%%%%%%%%%%%%%%%%%%%%%%

\end{minipage} 
\hfill
\begin{minipage}[t]{0.73\textwidth} 

%%%%%%%%%%%%%%%%%%%%%%%%%%%%%%%%%%%%%%
%     EXPERIENCE
%%%%%%%%%%%%%%%%%%%%%%%%%%%%%%%%%%%%%%

\section{Experience}
\runsubsection{Alibaba Summer of Code}
\descript{Student Participant}
\location{2019.06 - 2019.09 | Remote}
\vspace{\topsep}
\begin{tightemize}
    \item The first batch of ASoC, 21 out of 400+ applicants chosen to be a participant
    \item Implement Arduino framework and standard library support for AliOS-Things
\end{tightemize}
\sectionsep

\runsubsection{Google Summer of Code}
\descript{Student Participant}
\location{2020.06 - 2020.08 | Remote}
\begin{tightemize}
    \item Port FreeRTOS to AMP board Portenta H7, and Arduino Framework based on FreeRTOS
    \item \href{https://github.com/MRNIU/FreeRTOS-PortentaH7}{FreeRTOS-PortentaH7} Port FreeRTOS to Portenta H7
    \item \href{https://github.com/MRNIU/ArduinoCore-freertos}{ArduinoCore-freertos} Arduino Framework based on FreeRTOS
\end{tightemize}
\sectionsep

\runsubsection{Open Source Promotion Plan 2020, 2021}
\descript{Mentor}
\location{2020.05 - 2020.09, 2021.05 - 2021.09 | Remote}
\begin{tightemize}
    \item Organized by the Institute of Software Chinese Academy of Sciences and openEuler community, 
            help students complete the development
    \item Kernel: \href{https://github.com/Simple-XX/SimpleKernel}{SimpleKernel}
    \item Compiler: \href{https://github.com/Simple-XX/SimpleCompiler}{SimpleCompiler}
    \item Renderer: \href{https://github.com/Simple-XX/SimpleRenderer}{SimpleRenderer}
\end{tightemize}
\sectionsep

\runsubsection{Google Summer of Code}
\descript{Student Participant}
\location{2022.06 - NOW | Remote}
\begin{tightemize}
\item ARM port and device tree support for \href{https://www.haiku-os.org}{Haiku}
\item \href{https://summerofcode.withgoogle.com/programs/2022/projects/y2L1QWf1}{GSoC2022 info}
\end{tightemize}
\sectionsep

\runsubsection{ThoughtWorks}
\descript{Development Intern}
\location{2020.09 - NOW | Xi'an, Chain}
\begin{tightemize}
\item Security and Systems Research department
\item The MVVM development of a brand smart watch
\item File monitoring system based on FUSE
\end{tightemize}
\sectionsep

%%%%%%%%%%%%%%%%%%%%%%%%%%%%%%%%%%%%%%
%     RESEARCH
%%%%%%%%%%%%%%%%%%%%%%%%%%%%%%%%%%%%%%

\section{Projects}
\runsubsection{\href{https://github.com/Simple-XX}{\bf Simple-XX Organization}}
\descript{Founder}
\location{2017.09}
\begin{tightemize}
    \item Help students learn low-level knowledge and write code
    \item Partner of Open Source Promotion Plan
    \item Multiple topics: CPU, kernel, compiler, renderer, etc.
    \item Earn 2k+ star
    \end{tightemize}
\sectionsep

\runsubsection{National Innovation and Entrepreneurship Training Program}
\descript{Owner}
\location{2019.12}
\begin{tightemize}
    \item Research on mathematical formula Recognition based on machine learning
    \end{tightemize}
\sectionsep

%%%%%%%%%%%%%%%%%%%%%%%%%%%%%%%%%%%%%%
%     OPEN SOURCE
%%%%%%%%%%%%%%%%%%%%%%%%%%%%%%%%%%%%%%

\section{Open Source Contributions}
\begin{tabular}{ll}
    \href{https://github.com/EugeneLiu/translationCSAPP}{EugeneLiu/translationCSAPP} & CMU 15-213 Course video translation plan, \\
        & participate in translation proofreading \\
    \href{https://github.com/xitu/gold-miner}{xitu/gold-miner} & Participate in the translation and proofreading of \\
        & English technical articles and official tensorflow documents \\
    \href{https://github.com/Homebrew/homebrew-core}{Homebrew/homebrew-core} & tool chain for kernel development \\
    More... & See \href{https://github.com/MRNIU}{Github/MRNIU}
\end{tabular}
\sectionsep

%%%%%%%%%%%%%%%%%%%%%%%%%%%%%%%%%%%%%%
%     AWARDS
%%%%%%%%%%%%%%%%%%%%%%%%%%%%%%%%%%%%%%

\section{Awards} 
\begin{tabular}{rll}
    2019         & 2nd Prize  & The 5th China International College Students' 'Internet+' \\
                 &            & Innovation and Entrepreneurship Competition(Shaanxi) \\
    2018	     & 2nd Prize  & The 12th iCAN International Contest of innovAtioN(China Finals) \\
\end{tabular}
\sectionsep

%%%%%%%%%%%%%%%%%%%%%%%%%%%%%%%%%%%%%%
%     PUBLICATIONS
%%%%%%%%%%%%%%%%%%%%%%%%%%%%%%%%%%%%%%

% \section{Publications} 
% \renewcommand\refname{\vskip -1.5cm} % Couldn't get this working from the .cls file
% \bibliographystyle{abbrv}
% \bibliography{publications}
% \nocite{*}

\end{minipage} 
\end{document}  \documentclass[]{article}
